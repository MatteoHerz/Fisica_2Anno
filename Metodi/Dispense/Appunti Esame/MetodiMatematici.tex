\documentclass[x11names]{article}
\usepackage{tikz}
\usepackage{pgfplots}
\usepackage{xcolor}
\usepackage{svg}
\usepackage{amsmath}
\usepackage{array}
\usepackage[skins]{tcolorbox}
\usepackage[version=4]{mhchem}
\usepackage[a4paper, total={6in, 10in}]{geometry}
%\usepackage{fouriernc}
\usepackage{xymtex}
\usepackage{textcomp}
\usepackage{eurosym}
\usepackage{mathrsfs}
\usepackage{float}
\usepackage{pst-all}
\usepackage{pst-3dplot}
\usepackage{leftindex}
\usepackage{verbatim}
\usepackage{import}
\usepackage{xifthen}
\usepackage{pdfpages}
\usepackage{transparent}
\usepackage{import}
\usepackage{pdfpages}
\usepackage{transparent}
\usepackage{amssymb}
%\usepackage{graphicx}


\definecolor{myblue}{RGB}{224, 245, 255} 
\definecolor{myred}{RGB}{234, 222, 255}
\definecolor{myorange}{RGB}{255, 102, 0}

% box
\newtcolorbox{es}[2][]{%
	enhanced,colback=white,colframe=black,coltitle=black,
	sharp corners,boxrule=0.4pt,
	fonttitle=\itshape,
	attach boxed title to top left={yshift=-0.5\baselineskip-0.4pt,xshift=2mm},
	boxed title style={tile,size=minimal,left=0.5mm,right=0.5mm,
		colback=white,before upper=\strut},
	title=#2,#1
}

% definizioni
\newtcolorbox{blues}[2][]{%
	enhanced,colback=myblue,colframe=black,coltitle=black,
	sharp corners,boxrule=0.4pt,
	attach boxed title to top left={yshift=-0.5\baselineskip-0.4pt,xshift=2mm},
	boxed title style={tile,size=minimal,left=0.5mm,right=0.5mm,
		colback=myblue,before upper=\strut},
	title=#2,#1
}

% teoremi
\newtcolorbox{redes}[2][]{%
	enhanced,colback=myred,colframe=black,coltitle=black,
	sharp corners,boxrule=0.4pt,
	fonttitle=\itshape,
	attach boxed title to top left={yshift=-0.5\baselineskip-0.4pt,xshift=2mm},
	boxed title style={tile,size=minimal,left=0.5mm,right=0.5mm,
		colback=myred,before upper=\strut},
	title=#2,#1
}


%% regole
\renewcommand*\contentsname{Indice}
\setcounter{tocdepth}{4}
\setcounter{secnumdepth}{2}
\pgfplotsset{compat=1.15}


\usetikzlibrary{arrows}


\title{}
\author{Matteo Herz}
\date{}



\begin{document}
	
\begin{titlepage}
   \begin{center}
       \vspace*{8cm}
        
       \textbf{\LARGE Metodi Matematici per la Fisica}

       \vspace{0.5cm}
	   \textit{\Large Riassunto del programma d'esame}\\
	   \vspace{0.5cm}
	   \textbf{\large Matteo Herz}

       \vfill
            
       
            
       \vspace{0.8cm}
     
       
            
       corso A\\
       Università degli studi di Torino, Torino\\
       Gennaio 2025\\
       
            
   \end{center}
\end{titlepage}

\tableofcontents
\newpage


%\begin{center}
%\fboxsep11pt
%\colorbox{myblue}{\begin{minipage}{5.75in}
%	\begin{blues}{Definizione: }
%	\end{blues}
%\end{minipage}}       
%\end{center}

%\begin{center}
%\fboxsep11pt
%\colorbox{myred}{\begin{minipage}{5.75in}
%	\begin{redes}{}
%	\subsubsection{Teorema:  }
%	\end{redes}
%\end{minipage}}        
%\end{center}


% DOCUMENT %


\section{Integrali}
\subsection{Integrali trigonometrici}
\subsection{Integrali sulla retta reale}

\section{Equazioni differenziali omogenee a paramentri non costanti}
\subsection{Soluzione nell'intorno di un punto regolare}
\subsection{Soluzione nell'intorno di una singolarità fuchsiana}

\section{Fuorier}
Data $L$ in $\mathbb{R}$ si definisce serie di Fourier di coefficienti $a_n \in \mathbb{C}$ la serie di funzioni

\[
	\sum_{n=-\infty}^{+\infty} a_n \frac{e^{ik_{n}x}}{\sqrt{L}} \quad k_n = \frac{2\pi}{L}nl
\]

\subsection{Serie di Fourier}
\subsection{Trasformata di Fourier}
\textbf{C.D.E.} Affinché $\mathscr{F}[f(x)]$ esista finita è \textbf{condizione sufficiente} che $f(x)$ sia \textbf{sommabile}. Inoltre il \textit{Teorema di Dirichlet} assicura come nel caso della serie di Fourier, la convergenza puntuale della trasformata a $f(x_0)$ nei punti in cui $f(x)$ è continua e a metà dei punti di discontinuità dove $f(x)$ presenta dei salti.\\
Dunque affinché $f(x)$ sia sommabile devono essere rispettate \textbf{due condizioni}:
\begin{align}
	&\lim_{x \to x_0}(x-x_0)f(x) = 0 \quad\quad \forall x_0 \in \, ]-\infty,+\infty[ \\
	&\lim_{x \to \pm\infty}xf(x) =  0
\end{align}
Più in generale però esistono funzioni non sommabili che ammettono trasformata di Fourier. \\Un esempio molto importante è la funzione:
\[
	f(x)=\frac{\sin(x)}{x} \quad\Longrightarrow\quad \mathscr{F}_{k}[f(x)] = \begin{cases}
		\sqrt{\frac{1}{2}}\quad |x|<1 \\
		\frac{1}{2}\sqrt{\frac{1}{2}}\quad |x|=1 \\
		0 \quad |x|>1 
	\end{cases} 
\] 


\subsection{Antitrasformata di Fourier}

\section{Laplace}
\subsection{Trasformata di Laplace}
Prendiamo $f(t), \;\; t\in R\;\,$ e $\;f(t)e^{\alpha t},\;\; \alpha \in R$ \\
Se $f(t)$ è un polinomio per $t \to +\infty$ $\Longrightarrow$ $\;f(t)e^{\alpha t} \to 0$, ma se $t \to -\infty$ la funzione esplode.\\Si introduce allora la $\theta(t)$, detta \textbf{Theta di Heaviside}. 

\[
	\theta(t) = \begin{cases}
		1 \quad\text{se}\;\;  t >1 \\
		0 \quad\text{se}\;\;  t <0
	\end{cases}
\]
\subsection{Antitraformata di Laplace}
\subsection{Equazioni differenziali - Metodo di Laplace}




\end{document}



 
